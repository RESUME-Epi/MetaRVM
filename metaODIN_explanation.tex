\documentclass{article}
\usepackage{amsmath}
\usepackage{amssymb}
\usepackage{graphicx}
\usepackage{hyperref}
\usepackage{algorithm}
\usepackage{algpseudocode}
\usepackage{listings}
\usepackage{xcolor}
\usepackage{booktabs}

\title{Mathematical Description of the MetaRVM Metapopulation Compartmental Disease Model}
\author{MetaRVM Development Team}
\date{\today}

\begin{document}

\maketitle

\begin{abstract}
This document provides a comprehensive mathematical description of the metapopulation compartmental disease model implemented in the MetaRVM R package. The model simulates the spread of a respiratory virus across multiple interacting subpopulations, accounting for population mobility, vaccination, and time-varying mixing patterns. We present the model structure, compartmental transitions, and key mathematical formulations underlying the simulation engine.
\end{abstract}

\section{Introduction}

The \texttt{meta\_sim} function in the MetaRVM package implements a metapopulation compartmental disease model that simulates the spread of respiratory viruses across multiple interacting subpopulations. The model is particularly useful for studying the impact of mobility patterns, vaccination strategies, and other interventions on epidemic dynamics.

\section{Model Structure}

The model divides each subpopulation into the following compartments:

\begin{itemize}
    \item $S_i$: Susceptible individuals in subpopulation $i$
    \item $E_i$: Exposed individuals in subpopulation $i$
    \item $I_{p,i}$: Presymptomatic infectious individuals in subpopulation $i$
    \item $I_{a,i}$: Asymptomatic infectious individuals in subpopulation $i$
    \item $I_{s,i}$: Symptomatic infectious individuals in subpopulation $i$
    \item $H_i$: Hospitalized individuals in subpopulation $i$
    \item $R_i$: Recovered individuals in subpopulation $i$
    \item $D_i$: Deceased individuals in subpopulation $i$
    \item $V_i$: Vaccinated individuals in subpopulation $i$
\end{itemize}

Additionally, the model tracks:
\begin{itemize}
    \item $P_i$: Total population size of subpopulation $i$
    \item $I_{all,i}$: All infectious individuals ($I_{p,i} + I_{a,i} + I_{s,i}$) in subpopulation $i$
    \item $cum\_V_i$: Cumulative number of vaccinated individuals in subpopulation $i$
    \item $mob\_pop_i$: Mobile population in subpopulation $i$ (individuals who can move between subpopulations)
\end{itemize}

\section{Mathematical Formulation}

\subsection{Compartmental Dynamics}

The model evolves in discrete time steps. Let $t$ denote the current time step, and $\Delta t$ denote the time increment. The state update equations for each compartment are:

\begin{align}
S_i(t+\Delta t) &= S_i(t) - n_{SE,i}(t) + n_{RS,i}(t) + n_{VS,i}(t) - n_{SV\_eff,i}(t) \\
E_i(t+\Delta t) &= E_i(t) + n_{SE,i}(t) - n_{EI,i}(t) + n_{VE,i}(t) \\
I_{p,i}(t+\Delta t) &= I_{p,i}(t) + n_{EI\_presymp,i}(t) - n_{preI\_symp,i}(t) \\
I_{a,i}(t+\Delta t) &= I_{a,i}(t) + n_{EI\_asymp,i}(t) - n_{I\_asymp\_R,i}(t) \\
I_{s,i}(t+\Delta t) &= I_{s,i}(t) + n_{preI\_symp,i}(t) - n_{I\_symp\_RH,i}(t) \\
I_{all,i}(t+\Delta t) &= I_{p,i}(t+\Delta t) + I_{a,i}(t+\Delta t) + I_{s,i}(t+\Delta t) \\
R_i(t+\Delta t) &= R_i(t) + n_{I\_asymp\_R,i}(t) + n_{I\_symp\_R,i}(t) + n_{HR,i}(t) - n_{RS,i}(t) \\
H_i(t+\Delta t) &= H_i(t) + n_{I\_symp\_H,i}(t) - n_{HR,i}(t) - n_{HD,i}(t) \\
D_i(t+\Delta t) &= D_i(t) + n_{HD,i}(t) \\
P_i(t+\Delta t) &= P_i(t) - n_{HD,i}(t) \\
V_i(t+\Delta t) &= V_i(t) - n_{VS,i}(t) - n_{VE,i}(t) + n_{SV\_eff,i}(t) \\
cum\_V_i(t+\Delta t) &= cum\_V_i(t) + n_{SV\_eff,i}(t) \\
mob\_pop_i(t+\Delta t) &= S_i(t+\Delta t) + E_i(t+\Delta t) + I_{all,i}(t+\Delta t) + R_i(t+\Delta t) + V_i(t+\Delta t)
\end{align}

where $n_{X,i}(t)$ represents the number of individuals transitioning between compartments in subpopulation $i$ at time $t$.

\subsection{Transition Probabilities}

The probabilities of transitioning between compartments in a time step $\Delta t$ are:

\begin{align}
p_{SE,i}(t) &= 1 - \exp(-\lambda_{i,i}(t) \cdot \Delta t) \\
p_{VE,i}(t) &= 1 - \exp(-\lambda_{v,i}(t) \cdot \Delta t) \\
p_{EI\_presymp,i}(t) &= 1 - \exp(-\frac{1}{d_e} \cdot \Delta t) \\
p_{preI\_symp,i}(t) &= 1 - \exp(-\frac{1}{d_p} \cdot \Delta t) \\
p_{I\_asymp\_R,i}(t) &= 1 - \exp(-\frac{1}{d_a} \cdot \Delta t) \\
p_{I\_symp\_RH,i}(t) &= 1 - \exp(-\frac{1}{d_s} \cdot \Delta t) \\
p_{HRD,i}(t) &= 1 - \exp(-\frac{1}{d_h} \cdot \Delta t) \\
p_{RS,i}(t) &= 1 - \exp(-\frac{1}{d_r} \cdot \Delta t) \\
p_{VS,i}(t) &= 1 - \exp(-\frac{1}{d_v} \cdot \Delta t)
\end{align}

where:
\begin{itemize}
    \item $\lambda_{i,i}(t)$ and $\lambda_{v,i}(t)$ are the forces of infection for susceptible and vaccinated individuals in subpopulation $i$
    \item $d_e$, $d_p$, $d_a$, $d_s$, $d_h$, $d_r$, and $d_v$ are the mean durations (in days) in the corresponding states
\end{itemize}

\subsection{Mobility and Mixing Patterns}

A key feature of the model is the incorporation of time-varying mobility and mixing patterns between subpopulations. The model uses four different mixing matrices:

\begin{itemize}
    \item $m_{wd,day}$: Mixing matrix for weekday daytime (6 am - 6 pm)
    \item $m_{wd,night}$: Mixing matrix for weekday nighttime (6 pm - 6 am)
    \item $m_{we,day}$: Mixing matrix for weekend daytime (6 am - 6 pm)
    \item $m_{we,night}$: Mixing matrix for weekend nighttime (6 pm - 6 am)
\end{itemize}

At each time step, the appropriate mixing matrix $m(t)$ is selected based on the day of the week and time of day:

\begin{equation}
m(t) = 
\begin{cases}
m_{we,day} & \text{if } t \bmod 7 \in \{0, 6\} \text{ and daytime} \\
m_{we,night} & \text{if } t \bmod 7 \in \{0, 6\} \text{ and nighttime} \\
m_{wd,day} & \text{if } t \bmod 7 \notin \{0, 6\} \text{ and daytime} \\
m_{wd,night} & \text{if } t \bmod 7 \notin \{0, 6\} \text{ and nighttime}
\end{cases}
\end{equation}

\subsection{Effective Population Mixing}

To model the movement and interaction of individuals between subpopulations, the model calculates effective population sizes and effective numbers of infectious individuals in each location:

\begin{align}
eff\_prod_{i,j}(t) &= m_{i,j}(t) \cdot mob\_pop_i(t) \\
P_{eff,j}(t) &= \sum_{i=1}^{N_{pop}} eff\_prod_{i,j}(t) \\
S\_eff\_prod_{i,j}(t) &= m_{i,j}(t) \cdot (S_i(t) - n_{SV\_eff,i}(t)) \\
V\_eff\_prod_{i,j}(t) &= m_{i,j}(t) \cdot V_i(t) \\
I\_eff\_prod_{i,j}(t) &= m_{i,j}(t) \cdot I_{all,i}(t) \\
I_{eff,j}(t) &= \sum_{i=1}^{N_{pop}} I\_eff\_prod_{i,j}(t)
\end{align}

where:
\begin{itemize}
    \item $eff\_prod_{i,j}(t)$ represents the effective number of individuals from subpopulation $i$ present in location $j$ at time $t$
    \item $P_{eff,j}(t)$ is the effective total population in location $j$ at time $t$
    \item $I_{eff,j}(t)$ is the effective number of infectious individuals in location $j$ at time $t$
\end{itemize}

\subsection{Force of Infection}

The force of infection for susceptible and vaccinated individuals in location $j$ is calculated as:

\begin{align}
\lambda_{i,j}(t) &= \beta_{i,j} \cdot \frac{I_{eff,j}(t)}{P_{eff,j}(t)} \\
\lambda_{v,j}(t) &= \beta_{v,j} \cdot \frac{I_{eff,j}(t)}{P_{eff,j}(t)}
\end{align}

where:
\begin{itemize}
    \item $\beta_{i,j}$ is the transmission rate for susceptible individuals in location $j$
    \item $\beta_{v,j}$ is the transmission rate for vaccinated individuals in location $j$
\end{itemize}

\subsection{Stochastic Transitions}

The model can be run in deterministic or stochastic mode. In stochastic mode, the number of individuals transitioning between compartments is drawn from binomial distributions:

\begin{align}
n_{SE\_eff,j,i}(t) &= 
\begin{cases}
0 & \text{if } S_i(t) \leq 0 \\
\text{Binomial}(S\_eff\_prod_{j,i}(t), p_{SE,i}(t)) & \text{if stochastic and } S_i(t) > 0 \\
S\_eff\_prod_{j,i}(t) \cdot p_{SE,i}(t) & \text{if deterministic and } S_i(t) > 0
\end{cases} \\
n_{SE,i}(t) &= \sum_{j=1}^{N_{pop}} n_{SE\_eff,j,i}(t)
\end{align}

Similar equations apply for other transitions between compartments.

\subsection{Vaccination Dynamics}

Vaccination is implemented through a time-varying vaccination matrix $vac(t, i)$ that specifies the number of individuals to be vaccinated in each subpopulation at each time step. The effective number of vaccinated individuals accounts for vaccine efficacy:

\begin{equation}
n_{SV\_eff,i}(t) = n_{SV,i}(t) \cdot ve_i
\end{equation}

where $ve_i$ is the vaccine efficacy in subpopulation $i$.

\section{Implementation Details}

The model is implemented using the \texttt{odin} package in R, which provides a domain-specific language for defining and solving differential equations. The \texttt{meta\_sim} function encapsulates the model definition, parameter initialization, and simulation execution.

\subsection{Key Parameters}

\begin{itemize}
    \item $N_{pop}$: Number of subpopulations
    \item $\beta_i$ (or $ts$): Transmission rate for susceptible individuals
    \item $\beta_v$ (or $tv$): Transmission rate for vaccinated individuals
    \item $d_e$, $d_p$, $d_a$, $d_s$, $d_h$, $d_r$, $d_v$: Mean durations in days for each compartmental state
    \item $pea$: Proportion of exposed individuals becoming asymptomatic
    \item $psr$: Proportion of symptomatic individuals recovering (vs. hospitalized)
    \item $phr$: Proportion of hospitalized individuals recovering (vs. dying)
    \item $ve$: Vaccine efficacy
\end{itemize}

\subsection{Checkpointing Functionality}

The model includes a checkpointing mechanism that saves the state of the simulation at the end of the run, allowing for resumption of the simulation from that point. This is particularly useful for long simulations or when incorporating real-time data updates.

\section{Model Applications}

This metapopulation model can be applied to various scenarios, including:

\begin{itemize}
    \item Studying the impact of mobility restrictions on disease spread
    \item Evaluating vaccination strategies across heterogeneous populations
    \item Analyzing the effects of time-varying mixing patterns on epidemic dynamics
    \item Forecasting disease burden in interconnected regions
    \item Testing intervention strategies in complex population structures
\end{itemize}

\section{Conclusion}

The metapopulation compartmental disease model implemented in the MetaRVM package provides a flexible and powerful framework for simulating the spread of respiratory viruses across interconnected subpopulations. By incorporating mobility patterns, vaccination dynamics, and stochastic effects, the model can capture the complex dynamics of real-world epidemics and help inform public health decision-making.

\end{document}